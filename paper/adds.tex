\chapter*{Приложения}
\addcontentsline{toc}{chapter}{Приложения}

Метод Симпсона
\begin{lstlisting}[language=C]
#ifndef SIMPSON_H_
#define SIMPSON_H_

#include "utils.h"
namespace{

template<typename F>
class SubSum {
public:
  SubSum(const F& f, double offset, double scale)
    : f_(f), offset_(offset), scale_(scale) {
  }

  double operator () (int i) const {
    return f_(offset_ + scale_ * i);
  }

private:
  const F& f_;
  double offset_;
  double scale_;
};
}

class Simpson{
  public:

  template <class F>
  static double Integrate(double from, 
                          double to, 
                          int steps, 
                          const F& f){
    double  h = (to - from) / steps;

    SubSum<F> even_sum(f, from + 2 * h, 2 * h);
    SubSum<F> odd_sum(f, from + h, 2 * h);

    double sum1 = utils::KahanSum((steps - 1) / 2, even_sum);
    double sum2 = utils::KahanSum(steps / 2, odd_sum);

    double sum = f(from) + 2*sum1 + 4*sum2 + f(to);
    sum *= h/3;
    return sum;
  }
};

#endif  // SIMPSON_H_

\end{lstlisting}

Метод Гаусса
\begin{lstlisting}[language=C]
#ifndef GAUSS_H_
#define GAUSS_H_

#include <cmath>

class Gaus{
  public:

  template <class F>
  static double Integrate(double from, double to, const F& f){
    double summand = (from+to)/2.0, sum, multiplier = (to-from)/2.0;
    sum =  128/255.0*f(summand)+
           (322+13*sqrt(70))/900.0*
             f(summand+1/3.0*sqrt(5-2*sqrt(10/7.0)))*multiplier +
           (322+13*sqrt(70))/900.0*
             f(summand-1/3.0*sqrt(5-2*sqrt(10/7.0)))*multiplier +
           (322-13*sqrt(70))/900.0*
             f(summand+1/3.0*sqrt(5+2*sqrt(10/7.0)))*multiplier +
           (322-13*sqrt(70))/900.0*
             f(summand-1/3.0*sqrt(5+2*sqrt(10/7.0)))*multiplier;

    return sum*(to-from)/2.0;
  }
};

#endif  // GAUSS_H_
\end{lstlisting}

Метод Кахана
\begin{lstlisting}[language=C]
  template<typename F>
  double KahanSum(int n, const F& f) {
    double s = 0, c = 0, t, y;
    for(int j=0; j<n; j++){
      y = f(j) - c;
      t = s + y;
      c = (t - s) -y;
      s = t;
    }
    return s;
  }
};

\end{lstlisting}
