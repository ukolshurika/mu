\section{Литературный обзор}

При рассмотрении задачи о стесненном деформировании мембраны выделяются 
два этапа деформирования: свободное деформирование (происходит вплоть до касания мембраной стенок матрицы) и стесненное деформирование (от касания мембраной стенок матрицы до заполнения или разрушения мембраны). До настоящего момента существует  не так много решений задач о деформировании мембраны.

\subsection{Задача свободного деформирования, при учете упрочнения материала}
Решение задачи о свободном деформировании мембраны шириной $2l$ и начальной толщиной $h_0$, закрепленной с двух длинных сторон и нагруженной равномерным давлением $q$, приведено в монографии Н.Н. Малинина [\ref{malinin}], где используется модель упрочнения материала:

\begin{equation}
	\sigma_e = A\xi_e^{m_1}\text{P}^{m_2},
\end{equation}

где $\sigma_e$~-- интенсивность напряжений, $\alpha$~-- половина угла раствора, $\xi_e $ деформации ползучести, $A, \; m_1\; m_2$~-- параметры 
материала, являющиеся справочной информацией[\ref{malinin_132}, \ref{malinin_136}], $P$~--- параметр Удквиста. 

Как результат решения задачи приводится зависимость угла раствора мембраны от безразмерного времени:
%\begin{equation}

	$$\int\limits^t_0q^{1/m_1}\,dt = b\int\limits_0^{\alpha}\Phi\,d\alpha;$$
	$$\Phi  = \left(\dfrac{\sin^2\alpha}{\alpha}\right)^{1/m_1}\left[\ln(\dfrac{\alpha}{\sin\alpha})\right]^{m_2/m_1}
	\left(\dfrac{1}{a} - \ctg \alpha \right);$$
	$$b = \left(\dfrac{2}{\sqrt 3}\right)^{(m_1+m_2+1)/m_1}(aH_0/l)^{1/m_1}$$
%\end{equation}

\subsection{Задача стесненного деформирования, при условии упрочнения материала}
Рассмотрим решение задачи, в случае идеального скольжения мембраны вдоль стенок матрицы, приведенное в 
монографии Н.Н. Малинина [\ref{malinin}]:

%\begin{equation}

	$$\int\limits^{t_1}_tq^{1/m_1}\,dt = b\int\limits_0^x\Phi_1\,d\alpha;$$
	$$\Phi_1  = \dfrac{\chi_1}{\chi_2+\chi_1x}[\ln(\chi_2+\chi_1x)]^{\frac{m_2}{m_1}}
	\left\lbrace\dfrac{\alpha}{[\chi_2-(1-\chi_1)x](\chi_2+\chi_1x)}\right\rbrace^\frac{1}{m_1};$$
	$$b = \left(\dfrac{2}{\sqrt 3}\right)^{(m_1+m_2+1)/m_1}(aH_0/l)^{1/m_1};$$
	$$\chi_1=1-\alpha\ctg\alpha;\;\chi_2=\alpha\sin\alpha.$$
%\end{equation}

Из этого уравнения численно определяется зависимость безразмерной длины контакта от времени $x(t)$ для 
заданного закона изменения давления во времени. Зная длину участка контакта, можно определить толщину 
мембраны и напряжения. Решения для случая прилипания также приведено в [\ref{malinin}].

\subsection{Задача стесненного деформирования внутри криволинейной матрицы}
В статье [\ref{jerebcov}] приводится решение задачи деформирования мембраны шириной $2a$ и начальной толщиной $H_0$ внутри криволинейной матрицы.
Мембрана нагружена односторонним равномерным давлением $q = q(t)$, которое может изменяться во времени по определенному закону.
Для решения задачи была взята теория упрочнения:

\begin{equation}
p^{\gamma}_e\dfrac{dp_e}{dt} = A\sigma_e^n,
\end{equation} 

где $\sigma_e$ и $p_e$~-- интенсивности напряжений и деформаций, $n, \gamma, A$~--- параметры материала при заданной температуре.
Для этапа стесненного деформирования было получено соотношение (для случая идеального скольжения мембраны о стенки матрицы): 

\begin{equation}
t^{n+1} = t_1^{n+1}+\left(\dfrac{2}{\sqrt 3}\right)^{n+1}(n+1)(q)^{-n}\int\limits_1^{x_0}\dfrac{(\overline{\rho} d\alpha + \alpha 
   d\overline{\rho}+d\overline{s})}{(\overline{\rho} \alpha + \overline{s})}\left(\dfrac{H}{\rho}\right)^n\, dx_0
\end{equation}

В данной статье накладывается ограничение на профиль матрицы: кривизна кривой, описывающей профиль
матрицы должна монотонно увеличиваться от точки $a$ к $0$.
 В качестве примера рассматривается профиль матрицы заданный параболой: $y= b(1-x^k), 1<k\leqslant 2$. 

\subsection{Задача стесненного деформирования при условии установившейся ползучести}
Рассматривается деформирование мембраны шириной $2a$ и начально толщиной $H_0$, закрепленной вдоль длинных сторон и нагруженной равномерным поперечным давлением $q$.
В работах [\ref{teraud_dis}, \ref{teraud}] рассматривается решение, основанное на дробно-сингулярной модели установившейся ползучести [\ref{shest}]:

 \begin{equation}
  \dot{p}_u = C \left( \frac{\sigma_u}{\sigma_b - \sigma_u} \right)^n, 
  \label{main_equation}
 \end{equation}
в которой $\sigma_u$ и $\textit{\.{p}}_u$~--- интенсивности напряжений и скоростей деформации ползучести, $\sigma_b$~--- предел кратковременной прочности материала, $C$ и $n$~--- постоянные.


 Известные работы [\ref{malinin},\ref{jerebcov}], описанные выше, допускают появление нефизичных бесконечных напряжений ($\sigma_u \to \infty$) в начальный момент времени. Для их исключения в данной работе и [\ref{teraud_dis}, \ref{teraud}] дополнительно учитывается мгновенное деформирование в качестве отдельного этапа.
 
 Основное соотношение стадии свободного деформирования, выраженное в безразмерных переменных:
 \begin{equation}
	\overline{q} = \dfrac{q}{\sigma_b}, \overline{H} = \dfrac{H}{H_0}, \overline{H}_0 = \dfrac{H_0}{a}, \overline{t} = \dfrac{\sqrt 3}{2}Ct, 
 \end{equation}   

принимает вид:
\begin{equation}
t=\int\limits_{\alpha_1}^{\alpha}\left(\dfrac{1}{\alpha}-\ctg \alpha\right)\left(\dfrac{2H_0\sin^2\alpha}{\sqrt 3 q \cdot \alpha}-1\right)^n \, d\alpha
\end{equation}   

В работах [\ref{teraud_dis}, \ref{teraud}] стадия деформирования решена с помощью конечно разностных схем, при условии трения 
мембраны о стенки матрицы. Идеальное скольжение получается из этого решения путем обнуления коэффициента трения.
Основные формулы в кратком обзоре опустим.

