\documentclass[a4paper,12pt, openany]{memoir}
\usepackage{msiu_term_work}
\usepackage{my_term_styles}
% \usepackage{samgtubib}
\usepackage{lastpage}
\usepackage{amsmath}
\usepackage{xcolor}
\usepackage{multirow}
\usepackage{caption}
\usepackage{chngcntr}

\numberwithin{equation}{chapter}
\renewcommand{\theequation}{\arabic{chapter}.\arabic{equation}}

\setcounter{figure}{0}
\numberwithin{figure}{chapter}
\renewcommand{\thefigure}{\arabic{chapter}.\arabic{\c@figure}}
\captionsetup{width=15cm}
\begin{document}

\begin{center}
Локощенко А.М., Уколова А.В.

{\textbf{Деформирование длинной узкой прямоугольной мембраны в стесненных условиях}}
\end{center}
%\tableofcontents

Исследуется деформирование длинной узкой прямоугольной мембраны внутри криволинейной матрицы под действием равномерного поперечного
давления $q$. При этом предполагаются закрепление мембраны вдоль ее длинных сторон и трение поверхности мембраны относительно поверхности матрицы.

Для описания деформирования мембраны предлагается сингулярная дробно-степенная модель установившейся ползучести материала [\ref{uav14:shest}]


\begin{equation}
\textit{\.{p}}_u = C \left( \frac{\sigma_u}{\sigma_b - \sigma_u} \right)^n,
\label{uav14:f1}
\end{equation}

в которой $\sigma_u$ и $\textit{\.{p}}_u$~--- интенсивности напряжений и скоростей деформации ползучести, $\sigma_b$~--- предел кратковременной прочности материала, $C$ и $n$~--- постоянные.

Решение задачи о деформировании мембраны в стесненных условиях при учете упрочнения материала приведены в монографии Н.Н.~Малинина [\ref{uav14:mal}]. Известные работы [\ref{uav14:mal},\ref{uav14:jer}] допускают появление нефизичных бесконечных напряжений ($\sigma_u \to \infty$) в начальный момент времени, для их исключения в данной работе дополнительно учитывается мгновенное деформирование. В [\ref{uav14:ter}] приведено решение рассматриваемой задачи при различных граничных условиях, однако только для клиновидной матрицы. В [\ref{uav14:samara}] приведено решение рассматриваемой задачи только для случая идеального скольжения. В данной работе приводится обобщение результатов, полученных в [\ref{uav14:samara}, \ref{uav14:ter}], на случай криволинейной матрицы при наличии трения. Напряженное состояние мембраны можно считать безмоментным. Поскольку длина мембраны значительно превосходит её ширину, можно считать, что реализуется случай плоской деформации.

Решение данной задачи проводится в трех последовательных стадиях. Для исключения бесконечных напряжений в начальный момент времени $t$ (первая стадия) предполагается, что мембрана обладает упругими свойствами. Упругое деформирование мембраны описывается с помощью
закона Гука при сложном напряженном состоянии при учете несжимаемости материала мембраны. На второй стадии (стадии свободного деформирования) моделируется ползучесть мембраны вплоть до её касания стенок матрицы. Решение задачи на первых двух стадиях полностью совпадает с решением,
приведенным в [\ref{uav14:ter_dis}, \ref{uav14:ter}]. Исследование ползучести мембраны на третьей стадии в данной постановке выполнено впервые.

Началом стесненного деформирования (третья стадия) считается момент времени, при котором мембрана впервые касается матрицы. В [\ref{uav14:jer}] приведены уравнения, характеризующие деформированное состояние мембраны.

Примем безразмерные величины:
 $$\overline{q} = \dfrac{q}{\sigma_b}, \overline{H} = \dfrac{H}{H_0}, \overline{H}_0 = \dfrac{H_0}{a}, \overline{t} = \dfrac{\sqrt 3}{2}Ct$$
где $H$ ~--- толщина мембраны в произвольный момент времени, $H_0$~--- толщина мембраны при $t = -0$, $q$~--- давление, $2a$~--- ширина. При исследовании этой стадии рассматривается трение мембраны о стенки матрицы, поверхность которой задана уравнением $\overline{y} = f(\overline{x}) = \overline{b}(1-(\overline{x})^k)$, $x$ и $y$ характерные размеры матрицы(см. рис. 1), $\overline{x} = \frac x a$, $\overline{y} = \frac y a$,$b$~--- глубина матрицы (рис. 1) $\overline{b} = \frac b a$. При анализе деформирования мембраны всюду черточки над безразмерными переменными опущены.

 Моделирование деформирования мембраны на третьей стадии решается итерационным методом. Применяется аппроксимация производных по первому порядку точности. Произвольная $(k+1)$-ая итерация характеризуется приращением времени $dt^{k+1}$ и соответствующим этому приращению дополнительным участком свободной части мембраны $ds^{k+1}_{k+1}$, который начинает контактировать со стенками матрицы (рис.\ref{uav14:pic1}). Нижние цифровые индексы при параметрах обозначают шаг, на котором произошло прилегание этого участка к матрице, а верхние индексы~---шаг, на котором берутся эти параметры во времени. На границе мембраны и матрицы принимается закон трения Кулона с коэффициентом трения $\mu$. При этом известны следующие значения параметров, полученные на предыдущих шагах: $\rho_0^k,\ldots,\rho_k^k$; $\alpha_0^k,\ldots,\alpha_k^k$; $ds_0^k,\ldots,ds_k^k$; $(\sigma_\theta)_0^k,\ldots,(\sigma_\theta)_k^k$; $(\dot{p}_\theta)_0^k,\ldots,(\dot{p}_\theta)_k^k$.
 
 Зададим $dt^{k+1}$. На $(k+1)$-ом шаге необходимо вычислить $ds^{k+1}_{k+1},\; (\sigma_\theta)_{k+1}^{k+1},\;H_{k+1}^{k+1}$ и найти новые значения
 системы рассматриваемых величин.
 
 Соотношения, описывающие удлинение участка касания мембраны приведены в [\ref{uav14:ter}].
 
 Дуга $\alpha_k^k\rho_k^k$ за время $dt^{k+1}$ переходит в дугу $(\alpha_k^k+d\alpha_{k+1}^k)(\rho_k^k+d\rho_{k+1}^k) + ds_{k+1}^{k+1})$, следовательно
 приращение окружной деформации равно:
\begin{equation}
\d(p_\theta)^{k+1}_{k+1} = \dfrac{(\alpha_k^k+d\alpha_{k+1}^k)(\rho_k^k+d\rho_{k+1}^k) + ds_{k+1}^{k+1}) - \alpha_k^k\rho_k^k}{\alpha_k^k\rho_k^k}
\label{uav14:f1.5}
\end{equation}

Очевидно, что длина дуги касания, определенная вплоть до каждого участка равна
\begin{equation}
s^{k+1}_i = \sum\limits^{k+1}_{i=0} ds_i^{k+1}
\label{uav14:f2}
\end{equation}



Из геометрических соображений получим зависимость между координатой касания $x^{k+1}_i$ мембраны и длиной дуги касания $s_i^{k+1}$:

\begin{equation}
s_i^{k+1}(x_i^{k+1}) = \int\limits^1_{x_i^{k+1}}\sqrt{1+f_x'^2}\,dx;
\label{uav14:f3}
\end{equation}


Зная длину дуги из формулы (\ref{uav14:f1}) и соотношение (\ref{uav14:f2}) получаем в неявном виде зависимость $x^{k+1}_i(s_i^{k+1})$.
Зная координаты касания участков мембраны из геометрических соображений найдем величины $\rho_{k+1}^{k+1},\;d\rho_k^{k+1},\alpha_{k+1}^{k+1},d\alpha_k^{k+1}$:

\begin{eqnarray}
\left.
\begin{split}
\rho(x^{k+1}_i) &= \sqrt{(y^{k+1}_i-(y^{k+1}_i)_c)^2 + (x^{k+1}_i)^2};\; d\rho_k^{k+1} = \rho'_{x^{k+1}_i}dx^{k+1}_i\\
\alpha(x^{k+1}_i)&= \dfrac{\pi}{2} - \arctg(g'_{x^{k+1}_i});\; d\alpha_k^{k+1} = \alpha_k^k'_{x^{k+1}_i}dx^{k+1}_i
\end{split}
\right\}
\label{uav14:f3.5}
\end{eqnarray}

Вырезая элемент мембраны и записывая уравнения его равновесия в проекцияах на нормаль и касательную получаем \ref{uav14:mal}:

\begin{equation}
\sigma_\theta = \dfrac{q\rho}{H}, \; \sigma_\theta H = \texttt{const}.
\label{uav14:f4}
\end{equation}

Из уравнения равновесия (\ref{uav14:f4}), подставленного в (\ref{uav14:f1}) получаем:

\begin{equation}
(\dot{p}_\theta)^{k+1}_{k+1} = \left(\dfrac{1}{1-\dfrac{\sqrt{3}}{2}\dfrac{\rho_k^kq}{H_0H^h_k}}-1\right)^n.
\label{uav14:f5}
\end{equation}

Исключая $(\dot{p}_\theta)^{k+1}_{k+1}$ из уравнения (\ref{uav14:f5}) при помощи (\ref{uav14:f1.5}) получим:
\begin{equation}
ds_{k+1}^{k+1} = \left(\dfrac{1}{1-\dfrac{\sqrt{3}}{2}\dfrac{\rho_k^kq}{H_0H^h_k}}-1\right)^n dt - \alpha_k^k*d\rho_k^{k+1}- \rho_k^k*d\alpha^{k+1}_k-d\rho^{k+1}_k*d\alpha_k^{k+1},
\label{uav14:f6}
\end{equation}
 где $\rho^k_k,\; \alpha^k_k, \; d\alpha^{k+1}_k,\; d\rho_k^{k+1} $ находятся по соотношениям (\ref{uav14:f3.5}).

 Так как радиальная скорость деформации $\dot{p_t} = -\dot{p_\theta}={\dot{H}}/{H}$, то записывая производную 
 $\dot{H}$ в виде конечных разностей, получаем соотношение, из которого определяются новые значения толщины участков мембраны:
 \begin{equation}
 H^{k+1}_i = H^k_i + dH^{k+1}_i = H_i^k\left[1-\left(\dfrac{1}{1-\dfrac{\sqrt{3}}{2}\dfrac{\rho_k^kq}{H_0H^h_k}}-1\right)^n\right]
\label{uav14:f7}
\end{equation}

Новые значения напряжений $\sigma_\theta$ вычисляются из уравнения равновесия элемента $A_i^kA_{i+1}^k$ (рис. \ref{uav14:schema}). Вывод соотношений с учетом закона трения кулона предствалены в работе [] \ref{uav14:ter}.

 \begin{equation}
 (\sigma_\theta)^{k+1}_{k+1} = \dfrac{\rho^{k+1}_{k+1}q}{H_0H^{k+1}_{k+1}}, \; (\sigma_\theta)^{k+1}_i =
 \dfrac{(\sigma_\theta)^{k+1}_{i+1}H^{k+1}_{i+1}}{H^{k+1}_{i}} - \mu\dfrac{ds^{k+1}_{i+1}q}{H_0H^{k+1}_i}
\label{uav14:f8}
\end{equation}

\begin{figure}[h!]
\begin{minipage}[m]{0.27\linewidth}
\vspace*{10pt}	
\center{\includegraphics[width=1.5\linewidth]{images/fig1.eps}}
  \label{uav14:pic1}
  \caption{ }
\end{minipage}
\hfill
\hspace*{20pt}
\begin{minipage}[h]{0.7\linewidth}
\center{\includegraphics[width=0.8\linewidth]{images/fig0.eps}}
  \label{uav14:pic}
  \caption{ }
\end{minipage}


\end{figure}

\Section[n]{Результаты расчетов}
В качестве примера рассмотрим деформирование мембраны из алюминиевого сплава Д16T при $400^\circ\text{C}$ [\ref{uav14:ter}]. Константы материала:
   $C=9.37\cdot10^5 \text{МПа}^{-n}\text{сек}^{-1}$, $n=3.4$, $\sigma_b = 88.3\; \text{МПа}$.
   Геометрические размеры мембраны: ширина $2a=200$ мм, толщина $H_0=2$ мм,$k$=1.5, $\overline{b}$=4.5 давление $q=2.65$ кПа [\ref{uav14:ter}].
   
   Вычисления показали, что мембрана в трения полностью заполняет матрицу за время $t = $ при $\mu=0.1$ и разрушается в ходе третьей стадии при $\mu=0.1$ в момент времени $t = $.
   Стадия мгновенного деформирования характеризуется параметром $\overline{H}_1 = 0.97$,
   стадия свободного деформирования характеризуется параметрами $\overline{H}_2 = 0.69, \overline{t}_2 = 5.36 \cdot 10^8$.
  
  На рис.2 представлен график зависимости толщины мембраны и интенсивности напряжения от времени
    (кривые 1, 2 и 3, 4 соответственно для $\mu=0.1$ и $\mu=0.5$).

\smallskip
Работа выполнена при поддержке РФФИ (проект \No 11--08--00008).

\begin{verbatim}
loko@imec.msu.ru; ukolshurika@gmail.com
\end{verbatim}	


% \begin{thebibliography}{9}
% \RBibitem{uav14:shest}\label{uav14:shest}
% \by Шестериков~С.\,А., Юмашева~М.\,А.
% \paper Конкретизация в уравнения состояния в теории ползучести.
% \jour Механика твердого тела
% \publ Машиностроение
% \yr 1984
% \issue 1
% \pages 86-91


% \RBibitem{uav14:mal}\label{uav14:mal}
% \by Малинин~H.\,H.
% \book Ползучесть в обработке металлов
% \publaddr М.
% \publ Машиностроение
% \yr 1986
% \totalpages 224

% \RBibitem{uav14:jer}\label{uav14:jer}
% \by Демин~В.\,А., Локощенко~А.\,М, Жеребцов~А.\,А.
% \paper Ползучесть длинной прямоугольной мембраны внутри криволинейной матрицы.
% \jour Известия ВУЗов. Машиностроение
% \yr 1998
% \issue 4-6
% \pages 41-46


% \RBibitem{uav14:ter_dis}\label{uav14:ter_dis}
% \by Терауд~В.\,В.
% \book Математическое моделирование технологических процессов, протекающих в условиях ползучести
% \publ Диссертация на соискание ученой степени кандидата технических наук
% \publaddr М. МГИУ
% \yr 2011
% \totalpages 210

% \RBibitem{uav14:ter}\label{uav14:ter}
% \by Локощенко~А.\,М. Терауд~В.\,В.
% \paper Ползучесть длинной узкой мембраны в стесненных условиях вплоть до разрушения.
% \jour Прикладная механика и техническая физика
% \yr 2013
% \issue 4 (принято к печати)


% \RBibitem{uav14:samara}\label{uav14:samara}
% \by Локощенко~А.\,М. Уколова~А.\,В.
% \paper Ползучесть длинной узкой мембраны внутри криволинейной матрицы.
% \jour Сборник трудов XI Всероссийской конференции <<Математическое моделирование и краевые задачи>>
% \yr 2013

% \end{thebibliography}
%\bibliographystyle{plainnat}
%\bibliography{bib}
%\printbibliography

\end{document}


\endinput